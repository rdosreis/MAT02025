% Options for packages loaded elsewhere
\PassOptionsToPackage{unicode}{hyperref}
\PassOptionsToPackage{hyphens}{url}
%
\documentclass[
  10pt,
  ignorenonframetext,
]{beamer}
\usepackage{pgfpages}
\setbeamertemplate{caption}[numbered]
\setbeamertemplate{caption label separator}{: }
\setbeamercolor{caption name}{fg=normal text.fg}
\beamertemplatenavigationsymbolsempty
% Prevent slide breaks in the middle of a paragraph
\widowpenalties 1 10000
\raggedbottom
\setbeamertemplate{part page}{
  \centering
  \begin{beamercolorbox}[sep=16pt,center]{part title}
    \usebeamerfont{part title}\insertpart\par
  \end{beamercolorbox}
}
\setbeamertemplate{section page}{
  \centering
  \begin{beamercolorbox}[sep=12pt,center]{part title}
    \usebeamerfont{section title}\insertsection\par
  \end{beamercolorbox}
}
\setbeamertemplate{subsection page}{
  \centering
  \begin{beamercolorbox}[sep=8pt,center]{part title}
    \usebeamerfont{subsection title}\insertsubsection\par
  \end{beamercolorbox}
}
\AtBeginPart{
  \frame{\partpage}
}
\AtBeginSection{
  \ifbibliography
  \else
    \frame{\sectionpage}
  \fi
}
\AtBeginSubsection{
  \frame{\subsectionpage}
}
\usepackage{amsmath,amssymb}
\usepackage{lmodern}
\usepackage{ifxetex,ifluatex}
\ifnum 0\ifxetex 1\fi\ifluatex 1\fi=0 % if pdftex
  \usepackage[T1]{fontenc}
  \usepackage[utf8]{inputenc}
  \usepackage{textcomp} % provide euro and other symbols
\else % if luatex or xetex
  \usepackage{unicode-math}
  \defaultfontfeatures{Scale=MatchLowercase}
  \defaultfontfeatures[\rmfamily]{Ligatures=TeX,Scale=1}
\fi
\usetheme[]{Antibes}
\usecolortheme{dolphin}
\usefonttheme{structurebold}
% Use upquote if available, for straight quotes in verbatim environments
\IfFileExists{upquote.sty}{\usepackage{upquote}}{}
\IfFileExists{microtype.sty}{% use microtype if available
  \usepackage[]{microtype}
  \UseMicrotypeSet[protrusion]{basicmath} % disable protrusion for tt fonts
}{}
\makeatletter
\@ifundefined{KOMAClassName}{% if non-KOMA class
  \IfFileExists{parskip.sty}{%
    \usepackage{parskip}
  }{% else
    \setlength{\parindent}{0pt}
    \setlength{\parskip}{6pt plus 2pt minus 1pt}}
}{% if KOMA class
  \KOMAoptions{parskip=half}}
\makeatother
\usepackage{xcolor}
\IfFileExists{xurl.sty}{\usepackage{xurl}}{} % add URL line breaks if available
\IfFileExists{bookmark.sty}{\usepackage{bookmark}}{\usepackage{hyperref}}
\hypersetup{
  pdftitle={MAT02025 - Amostragem 1},
  hidelinks,
  pdfcreator={LaTeX via pandoc}}
\urlstyle{same} % disable monospaced font for URLs
\newif\ifbibliography
\usepackage{longtable,booktabs,array}
\usepackage{calc} % for calculating minipage widths
\usepackage{caption}
% Make caption package work with longtable
\makeatletter
\def\fnum@table{\tablename~\thetable}
\makeatother
\setlength{\emergencystretch}{3em} % prevent overfull lines
\providecommand{\tightlist}{%
  \setlength{\itemsep}{0pt}\setlength{\parskip}{0pt}}
\setcounter{secnumdepth}{-\maxdimen} % remove section numbering
\usepackage{multirow}
\usepackage{xcolor}
\usepackage{booktabs}
\usepackage{longtable}
\usepackage{array}
\usepackage{multirow}
\usepackage{wrapfig}
\usepackage{float}
\usepackage{colortbl}
\usepackage{pdflscape}
\usepackage{tabu}
\usepackage{threeparttable}
\newcount\tmpnum
\def\tallymarks#1{\leavevmode \lower1bp\vbox to9bp{}%
   \tmpnum=#1
   \loop \ifnum\tmpnum<5 \kern1bp \tallynum\tmpnum \else \tallyV \fi
         \advance\tmpnum by-5
         \ifnum\tmpnum>0 \repeat
}
\def\tallynum#1{\bgroup\tmpnum=#1\relax
   \loop \ifnum\tmpnum>0
         \kern1bp \tallyI \kern1bp
         \advance\tmpnum by-1
         \repeat
   \egroup
}
\def\tallyI{\pdfliteral{q .5 w 0 -1 m 0 8 l S Q}}
\def\tallyV{\kern1bp\pdfliteral{q .5 w -1 0 m 9 7 l S Q}\tallynum4\kern1bp }

\setbeamertemplate{frametitle continuation}{}
\setbeamertemplate{items}[default] 
\setbeamertemplate{enumerate items}[default]
\setbeamertemplate{footline}{}
\setbeamercolor{alerted text}{fg=purple}
\titlegraphic{\hfill\includegraphics[height=1cm]{logos/Logo-40-anos-estatistica.png}}

\def\Corr{\mbox{Corr}\,}
\def\Cov{\mbox{Cov}\,}
\def\Var{\mbox{Var}\,}
\def\E{\mbox{E}\,}
\definecolor{darkpastelblue}{rgb}{0.47, 0.62, 0.8}
\definecolor{darkpastelgreen}{rgb}{0.01, 0.75, 0.24}
\definecolor{darkpastelpurple}{rgb}{0.59, 0.44, 0.84}
\definecolor{darkpastelred}{rgb}{0.76, 0.23, 0.13}

\ifluatex
  \usepackage{selnolig}  % disable illegal ligatures
\fi

\title{MAT02025 - Amostragem 1}
\subtitle{O erro quadrático médio}
\author{Rodrigo Citton P. dos Reis\\
\texttt{citton.padilha@ufrgs.br}}
\date{Porto Alegre, 2021}
\institute{\textsc{Universidade Federal do Rio Grande do Sul}\\
\textsc{Instituto de Matemática e Estatística}\\
\textsc{Departamento de Estatística}}

\begin{document}
\frame{\titlepage}

\hypertarget{o-erro-quadruxe1tico-muxe9dio}{%
\section{O erro quadrático médio}\label{o-erro-quadruxe1tico-muxe9dio}}

\begin{frame}[allowframebreaks]{O erro quadrático médio}
\protect\hypertarget{o-erro-quadruxe1tico-muxe9dio-1}{}
\begin{itemize}
\tightlist
\item
  Para comparar um estimador enviesado com um estimador imparcial, ou
  dois estimadores com diferentes valores de enviesamento, um critério
  útil é o \structure{erro quadrático médio} \textbf{(EQM)} da
  estimativa, medido a partir do valor da população que está sendo
  estimado.
\item
  Formalmente
\end{itemize}

\begin{eqnarray*}
EQM(\hat{\theta}) &=& \E(\hat{\theta} - \theta)^2 \\
&=& \E\{[\hat{\theta} - \textcolor{red}{\E(\hat{\theta})}] + [\textcolor{red}{\E(\hat{\theta})} - \theta]\}^2 \\
&=& \E\{[\hat{\theta} - \E(\hat{\theta})]^2 + 2[\hat{\theta} - \E(\hat{\theta})][\E(\hat{\theta}) - \theta] + [\E(\hat{\theta}) - \theta]^2\} \\
&=& \E\{[\hat{\theta} - \E(\hat{\theta})]^2\} + 2[\E(\hat{\theta}) - \theta]\E[\hat{\theta} - \E(\hat{\theta})] + \E[\E(\hat{\theta}) - \theta]^2 \\
&=& \Var(\hat{\theta}) + 2[\E(\hat{\theta}) - \theta]\textcolor{red}{[\E(\hat{\theta}) - \E(\hat{\theta})]} + [B(\hat{\theta})]^2 \\
&=& \Var(\hat{\theta}) + [B(\hat{\theta})]^2.
\end{eqnarray*}

\framebreak

\begin{itemize}
\tightlist
\item
  Note que se um estimador \(\hat{\theta}\) é não enviesado para
  \(\theta\), então
\end{itemize}

\[
EQM(\hat{\theta}) = \Var(\hat{\theta}) + [\textcolor{red}{0}]^2 = \Var(\hat{\theta}) = \sigma^2_{\hat{\theta}}.
\]

\begin{itemize}
\tightlist
\item
  No exemplo da aula passada, considerando o parâmetro populacional de
  interesse como a média, \(\mu\), temos
  \(EQM(\hat{\mu}) = \sigma^2_{\hat{\mu}} + B^2\), em que
  \(B = m - \mu\).
\end{itemize}

\framebreak

\begin{itemize}
\tightlist
\item
  O uso do EQM como critério de precisão de um estimador equivale a
  considerar equivalentes duas estimativas que têm o mesmo erro
  quadrático médio.
\item
  Isso não é inteiramente verdadeiro, porque a distribuição de
  frequência de erros \((\hat{\mu} - \mu)\) não será a mesma para dois
  estimadores, caso eles apresentem viéses de valores diferentes.
\item
  Entretanto,
  \structure{Hansen, Hurwitz e Madow (1953)}\footnote{Hansen, M. H., Hurwitz, W. N. e Madow, W. G. (1953) {\bf Sample Survey methods and theory}, John Wiley \& Sons, Nova York, Vol. I, pg. 58.}
  que se \(B/\sigma\) for menor que cerca que \(0,5\), as duas
  distribuições de frequência são quase idênticas em relação aos erros
  \textbf{absolutos} \(|\hat{\mu} - \mu|\) de tamanhos diferentes.
\end{itemize}
\end{frame}

\begin{frame}[allowframebreaks]{O erro quadrático médio}
\protect\hypertarget{o-erro-quadruxe1tico-muxe9dio-2}{}
\begin{itemize}
\tightlist
\item
  Mais uma vez suponha que \(\hat{\mu}\) tem uma distribuição
  aproximadamente normal com média \(m = \E(\hat{mu})\) e desvio padrão
  \(\sigma = \sigma_{\hat{\mu}}\). Ainda, denote
  \(EQM = EQM(\hat{\mu}) = \sigma^2 + B^2\)
\item
  Então
\end{itemize}

\[
\Pr\left(|\hat{\mu} - \mu| \geq  \sqrt{EQM}\right) = \frac{1}{\sigma\sqrt{2\pi}}\int_{\mu + \sqrt{EQM}}^{\infty}{e^{-(\hat{\mu} - m)^2/2\sigma^2}d\hat{\mu}}.
\]

\framebreak

\begin{longtable}[]{@{}cccc@{}}
\caption{Proporção de casos em que o valor verdadeiro, \(\mu\), não está
incluído no intervalo \(\hat{\mu} \pm k\sqrt{EQM}\), para diferentes
níveis de viés em \(\hat{\mu}\)}\tabularnewline
\toprule
\(B/\sigma\) & \(\geq \sqrt{EQM}\) & \(\geq 1,96\sqrt{EQM}\) &
\(\geq 2,576\sqrt{EQM}\) \\
\midrule
\endfirsthead
\toprule
\(B/\sigma\) & \(\geq \sqrt{EQM}\) & \(\geq 1,96\sqrt{EQM}\) &
\(\geq 2,576\sqrt{EQM}\) \\
\midrule
\endhead
0.0 & 0.32 & 0.050 & 0.0100 \\
0.1 & 0.32 & 0.050 & 0.0100 \\
0.2 & 0.32 & 0.050 & 0.0100 \\
0.3 & 0.32 & 0.050 & 0.0098 \\
0.4 & 0.32 & 0.050 & 0.0095 \\
0.5 & 0.32 & 0.049 & 0.0090 \\
0.6 & 0.32 & 0.048 & 0.0083 \\
1.0 & 0.35 & 0.038 & 0.0041 \\
1.5 & 0.38 & 0.021 & 0.0008 \\
2.0 & 0.41 & 0.009 & 0.0001 \\
2.5 & 0.42 & 0.003 & - \\
3.0 & 0.44 & 0.001 & - \\
\bottomrule
\end{longtable}

\framebreak

\begin{center}\includegraphics[width=1\linewidth]{C:/Users/rodri/OneDrive/Documentos/UFRGS/Disciplinas/Amostragem/MAT02025/output_pdf/05_eqm_files/figure-beamer/unnamed-chunk-2-1} \end{center}
\end{frame}

\begin{frame}[allowframebreaks]{Comentários}
\protect\hypertarget{comentuxe1rios}{}
\begin{itemize}
\tightlist
\item
  Esses resultados, para muitos propósitos práticos, concordam com as
  interpretações baseadas nos múltiplos correspondentes do desvio padrão
  quando uma estimativa não enviesada é usada.

  \begin{itemize}
  \tightlist
  \item
    Ou seja, quando
    \(\sqrt{EQM} = \sqrt{\sigma^2 + B^2} = \sqrt{\sigma^2 + 0} = \sigma\).
  \end{itemize}
\item
  Da aula passada, temos
\end{itemize}

\begin{center}\includegraphics[width=0.6\linewidth]{C:/Users/rodri/OneDrive/Documentos/UFRGS/Disciplinas/Amostragem/MAT02025/images/slide-aula-04} \end{center}

\framebreak

\scriptsize

\begin{itemize}
\tightlist
\item
  Devido à dificuldade de garantir que nenhum viés insuspeitado entre
  nas estimativas, geralmente falaremos da \structure{precisão} de uma
  estimativa em vez de sua \structure{acurácia} (exatidão).
\item
  A acurácia se refere ao tamanho dos desvios da verdadeira média
  \(\mu\), enquanto a precisão se refere ao tamanho dos desvios da média
  \(m\) obtida pela aplicação repetida do procedimento de amostragem.
\end{itemize}

\begin{center}\includegraphics[width=0.6\linewidth]{C:/Users/rodri/OneDrive/Documentos/UFRGS/Disciplinas/Amostragem/MAT02025/images/acuracia_precisao} \end{center}

\framebreak

\begin{center}\includegraphics[width=1\linewidth]{C:/Users/rodri/OneDrive/Documentos/UFRGS/Disciplinas/Amostragem/MAT02025/images/acuracia_precisao-2} \end{center}
\end{frame}

\hypertarget{exercuxedcios}{%
\section{Exercícios}\label{exercuxedcios}}

\begin{frame}{Para casa}
\protect\hypertarget{para-casa}{}
\begin{itemize}
\tightlist
\item
  \textbf{Atividade de avaliação I}.
\end{itemize}
\end{frame}

\begin{frame}{Próxima aula}
\protect\hypertarget{pruxf3xima-aula}{}
\begin{itemize}
\tightlist
\item
  Amostragem aleatória simples;
\item
  Definições e notação.
\end{itemize}
\end{frame}

\begin{frame}{Por hoje é só!}
\protect\hypertarget{por-hoje-uxe9-suxf3}{}
\begin{center}
{\bf Bons estudos!}
\end{center}

\begin{center}\includegraphics[width=0.5\linewidth,height=0.5\textheight]{C:/Users/rodri/OneDrive/Documentos/UFRGS/Disciplinas/Amostragem/MAT02025/images/Statistically-Insignificant-errorbar} \end{center}
\end{frame}

\end{document}
